\documentclass[11pt,a4paper]{article}
\usepackage[utf8]{inputenc}
\usepackage[T1]{fontenc}
\usepackage[french]{babel}
\usepackage{geometry}
\usepackage{booktabs}
\usepackage{longtable}
\usepackage{array}
\usepackage{xcolor}
\usepackage{colortbl}
\usepackage{graphicx}
\usepackage{fancyhdr}
\usepackage{titlesec}
\usepackage{tikz}
\usepackage{enumitem}
\usepackage{hyperref}
\usepackage{multicol}
\usepackage{listings}
\usepackage{tcolorbox}
\usepackage{amssymb}
\usepackage{float}
 

\geometry{margin=2cm}
\pagestyle{fancy}
\fancyhf{}
\rhead{Atelier Machine Learning}
\lhead{Analyse Comportementale Clientèle}
\rfoot{Page \thepage}

% Couleurs personnalisées
\definecolor{headerblue}{RGB}{41, 65, 114}
\definecolor{lightgray}{RGB}{245, 245, 245}
\definecolor{accentorange}{RGB}{230, 126, 34}
\definecolor{backcolour}{RGB}{248,248,248}

\tcbset{
    colback=backcolour,
    colframe=headerblue,
    fonttitle=\bfseries,
    boxrule=1pt,
    arc=3mm
}

\titleformat{\section}
{\color{headerblue}\normalfont\Large\bfseries}
{\thesection}{1em}{}[{\titlerule[2pt]}]

\titleformat{\subsection}
{\color{accentorange}\normalfont\large\bfseries}
{\thesubsection}{1em}{}

\begin{document}

\begin{titlepage}
\begin{tikzpicture}[remember picture,overlay]
    \fill[headerblue] (current page.north west) rectangle ([yshift=-5cm]current page.north east);
    \node[anchor=north west, xshift=2cm, yshift=-2cm, text=white] at (current page.north west) 
        {\Huge\textbf{Atelier Machine Learning}};
    \node[anchor=north west, xshift=2cm, yshift=-3.5cm, text=white] at (current page.north west) 
        {\Large\textit{Analyse Comportementale Clientèle Retail}};
\end{tikzpicture}

\vspace{6cm}

\begin{center}
{\LARGE\textbf{RAPPORT D'ANALYSE EXPLORATOIRE (EDA)}}\par
\vspace{1cm}
{\large Projet : E-commerce de Cadeaux}\par
\vspace{0.5cm}
{\large \textit{Chaîne complète : Exploration $\rightarrow$ Préparation $\rightarrow$ Modélisation $\rightarrow$ Évaluation $\rightarrow$ Déploiement}}\par
\vspace{2cm}
{\large \textbf{Préparé par : Islem Kaddoussi}}
\vspace{3cm}

\begin{tikzpicture}
    \draw[headerblue, line width=2pt] (0,0) -- (10,0);
\end{tikzpicture}

\vspace{2cm}
{\textbf{Année Universitaire : 2025-2026}}
\end{center}
\end{titlepage}

\tableofcontents
\newpage

\section{Introduction et Objectifs du Projet}

Ce projet s'inscrit dans une démarche complète de Machine Learning visant à transformer des données brutes en une application décisionnelle. Nous suivons le pipeline standard :      
\textbf{Exploration $\rightarrow$ Préparation $\rightarrow$ Modélisation $\rightarrow$ Évaluation $\rightarrow$ Déploiement.}

\subsection{Objectifs Pédagogiques}
Le tableau ci-dessous résume les compétences visées à chaque étape du projet :

\begin{table}[h]
\centering
\rowcolors{2}{lightgray}{white}
\begin{tabular}{|l|l|}
\hline
\textbf{Compétence} & \textbf{Description} \\ \hline
\textbf{Exploration} & Analyser la qualité et la structure des données (étape actuelle) \\ \hline
\textbf{Préparation} & Nettoyer, encoder et normaliser les features \\ \hline
\textbf{Transformation} & Réduire la dimension via l'Analyse en Composantes Principales (ACP) \\ \hline
\textbf{Modélisation} & Appliquer le clustering (K-Means), classification et régression \\ \hline
\textbf{Évaluation} & Interpréter les résultats et proposer des recommandations métiers \\ \hline
\textbf{Déploiement} & Créer une interface utilisateur interactive avec Flask \\ \hline
\end{tabular}
\caption{Pipeline du projet de Machine Learning}
\end{table}

\section{Description du Jeu de Données}
Le dataset contient des informations sur les clients et leurs transactions.
\begin{itemize}[label=\color{headerblue}$\bullet$]
    \item \textbf{Nombre de lignes :} 4372
    \item \textbf{Nombre de colonnes :} 52
    \item \textbf{Types de données :} Variables numériques (comportementales) et catégorielles (profils).
\end{itemize}

\section{Analyse de la Qualité des Données}

\subsection{Valeurs Manquantes}
L'analyse a révélé des lacunes importantes dans certaines variables clés :
\begin{itemize}
    \item \textbf{Age :} 1311 valeurs manquantes (nécessite une imputation par médiane ou mode).
    \item \textbf{AvgDaysBetweenPurchases :} 79 valeurs manquantes.
\end{itemize}


\subsection{Anomalies et Valeurs Suspectes}
Au-delà des valeurs nulles, l'analyse a révélé des codes erreurs ou des valeurs sentinelles (-1, 99, 999) :
\begin{itemize}[label=\color{accentorange}$\triangle$]
    \item \textbf{Quantités :} 349 valeurs à \textbf{-1} dans la colonne \textit{MinQuantity}.
    \item \textbf{Scores :} Des valeurs \textbf{-1} et \textbf{99} détectées dans \textit{SatisfactionScore}.
    \item \textbf{Support :} 87 valeurs à \textbf{999} dans \textit{SupportTicketsCount}.
    \item \textbf{Profils :} 1 649 clients avec un genre marqué comme \textit{"Unknown"}.
\end{itemize}

\section{Visualisations et Distributions}

\subsection{Histogrammes (Distributions)}

\begin{figure}[H]
\centering
\includegraphics[width=0.8\textwidth]{histograms.png}
\caption{Distribution des variables RFM et de l'âge}
\clearpage
\end{figure}

\subsection{Boxplots (Détection des Outliers)}

\begin{figure}[H]
\centering
\includegraphics[width=0.8\textwidth]{boxplots.png}
\caption{Analyse des valeurs aberrantes sur les variables monétaires}
\clearpage
\end{figure}
Les graphiques montrent des clients "extrêmes" avec des dépenses dépassant 10 000 unités. Ces outliers devront être traités pour ne pas biaiser le clustering.

\section{Analyse de Corrélation}
La matrice de corrélation montre des liens forts entre certaines variables (ex: \textit{MonetaryTotal} et \textit{Frequency}).
\begin{figure}[H]
\centering
\includegraphics[width=0.8\textwidth]{correlations.png}
\caption{Matrice de corrélation des caractéristiques numériques}
\clearpage
\end{figure}

\section{Stratégie de Modélisation (Clustering)}

\subsection{Justification de l'approche}
Bien qu'une colonne \textit{RFMSegment} existe déjà, nous allons appliquer un algorithme de \textbf{K-Means} pour :
\begin{itemize}
    \item Découvrir des groupes de clients plus précis en intégrant plus de variables (ex: Age, Ancienneté).
    \item Détecter les clients à haut risque de \textit{Churn} (attrition) cachés dans les segments loyaux.
\end{itemize}

\subsection{Variables de Cohorte et Ancienneté}
Nous avons converti \textit{RegistrationDate} en format date. Cela permettra de calculer la "Tenure" (ancienneté), variable cruciale car les clients inscrits depuis plus de 2 ans montrent une fréquence d'achat 30\% supérieure.

\section{Synthèse des Problèmes et Prochaines Étapes}

\begin{tcolorbox}[title=Bilan de la Qualité]
L'étape d'exploration confirme que le dataset est riche (52 colonnes) mais nécessite un nettoyage rigoureux :
\begin{itemize}
    \item Suppression du \textit{CustomerID} (inutile pour le modèle).
    \item Traitement des valeurs -1 et 999.
    \item Transformation logarithmique des variables monétaires très asymétriques.
\end{itemize}
\end{tcolorbox}

\textbf{Étape suivante :} Préparation des données (Nettoyage, Encodage des catégories et Scaling).

\end{document}

