\documentclass[11pt,a4paper]{article}
\usepackage[utf8]{inputenc}
\usepackage[T1]{fontenc}
\usepackage[french]{babel}
\usepackage{geometry}
\usepackage{booktabs}
\usepackage{array}
\usepackage{xcolor}
\usepackage{colortbl}
\usepackage{graphicx}
\usepackage{fancyhdr}
\usepackage{titlesec}
\usepackage{tikz}
\usepackage{enumitem}
\usepackage{hyperref}
\usepackage{multicol}
\usepackage{listings}
\usepackage{tcolorbox}
\usepackage{amssymb}
\usepackage{float}
\usepackage{amsmath}

\geometry{margin=2cm}
\pagestyle{fancy}
\fancyhf{}
\rhead{Atelier Machine Learning}
\lhead{Analyse Comportementale Clientèle}
\rfoot{Page \thepage}

% Couleurs personnalisées
\definecolor{headerblue}{RGB}{41, 65, 114}
\definecolor{lightgray}{RGB}{245, 245, 245}
\definecolor{accentorange}{RGB}{230, 126, 34}
\definecolor{backcolour}{RGB}{248,248,248}
\definecolor{warningred}{RGB}{220, 53, 69}

\tcbset{
    colback=backcolour,
    colframe=headerblue,
    fonttitle=\bfseries,
    boxrule=1pt,
    arc=3mm
}

\titleformat{\section}
{\color{headerblue}\normalfont\Large\bfseries}
{\thesection}{1em}{}[{\titlerule[2pt]}]

\titleformat{\subsection}
{\color{accentorange}\normalfont\large\bfseries}
{\thesubsection}{1em}{}

\begin{document}

% =============================================================================
% PAGE DE GARDE (TONNE)
% =============================================================================

\begin{titlepage}
\begin{tikzpicture}[remember picture,overlay]
    \fill[headerblue] (current page.north west) rectangle ([yshift=-5cm]current page.north east);
    \node[anchor=north west, xshift=2cm, yshift=-2cm, text=white] at (current page.north west) 
        {\Huge\textbf{Atelier Machine Learning}};
    \node[anchor=north west, xshift=2cm, yshift=-3.5cm, text=white] at (current page.north west) 
        {\Large\textit{Analyse Comportementale Clientèle Retail}};
\end{tikzpicture}

\vspace{6cm}

\begin{center}
{\LARGE\textbf{RAPPORT D'ANALYSE}}\par
\vspace{1cm}
{\large Projet : E-commerce de Cadeaux}\par
\vspace{0.5cm}
{\large \textit{Exploration $\rightarrow$ Préparation $\rightarrow$ Modélisation}}\par
\vspace{2cm}
{\large \textbf{Préparé par : [TON NOM]}}
\vspace{3cm}

\begin{tikzpicture}
    \draw[headerblue, line width=2pt] (0,0) -- (10,0);
\end{tikzpicture}

\vspace{2cm}
{\textbf{Année Universitaire : 2025-2026}}
\end{center}
\end{titlepage}

\tableofcontents
\newpage

% =============================================================================
% SECTION 1: INTRODUCTION (TONNE)
% =============================================================================

\section{Introduction et Objectifs du Projet}

Ce projet s'inscrit dans une démarche complète de Machine Learning visant à transformer des données brutes en une application décisionnelle. Nous suivons le pipeline standard :      
\textbf{Exploration $\rightarrow$ Préparation $\rightarrow$ Modélisation $\rightarrow$ Évaluation $\rightarrow$ Déploiement.}

\subsection{Objectifs Pédagogiques}

\begin{table}[h]
\centering
\rowcolors{2}{lightgray}{white}
\begin{tabular}{|l|l|}
\hline
\textbf{Compétence} & \textbf{Description} \\ \hline
\textbf{Exploration} & Analyser la qualité et la structure des données \\ \hline
\textbf{Préparation} & Nettoyer, encoder et normaliser les features \\ \hline
\textbf{Transformation} & Réduire la dimension via ACP \\ \hline
\textbf{Modélisation} & Appliquer le clustering (K-Means), classification et régression \\ \hline
\textbf{Évaluation} & Interpréter les résultats et proposer des recommandations \\ \hline
\textbf{Déploiement} & Créer une interface utilisateur avec Flask \\ \hline
\end{tabular}
\caption{Pipeline du projet de Machine Learning}
\end{table}

% =============================================================================
% SECTION 2: DONNÉES (FUSION)
% =============================================================================

\section{Description du Jeu de Données}

\begin{itemize}[label=\color{headerblue}$\bullet$]
    \item \textbf{Nombre de lignes :} 4 372 clients
    \item \textbf{Nombre de colonnes :} 52 features
    \item \textbf{Types de données :} 34 numériques, 18 catégorielles
    \item \textbf{Taille mémoire :} 1,7 MB
\end{itemize}

% =============================================================================
% SECTION 3: EDA (TONNE + COMPLÉMENTS)
% =============================================================================

\section{Analyse Exploratoire des Données (EDA)}

\subsection{Valeurs Manquantes}

\begin{tcolorbox}[colframe=warningred, title=Alertes]
\begin{itemize}
    \item \textbf{Age :} 1 311 valeurs manquantes (\textbf{30,0\%})
    \item \textbf{AvgDaysBetweenPurchases :} 79 valeurs manquantes (1,8\%)
\end{itemize}
\end{tcolorbox}

\subsection{Anomalies et Valeurs Suspectes}

\begin{itemize}[label=\color{accentorange}$\triangle$]
    \item \textbf{SatisfactionScore :} 115 valeurs \textbf{-1}, 114 valeurs \textbf{99}
    \item \textbf{SupportTicketsCount :} 43 valeurs \textbf{-1}, 87 valeurs \textbf{999}
    \item \textbf{Gender :} 1 649 clients \textbf{"Unknown"} (37,7\%)
    \item \textbf{Quantités négatives :} 349 valeurs \textbf{-1} dans MinQuantity, 15 dans TotalQuantity, 16 dans AvgQuantityPerTransaction
    \item \textbf{Codes 99 :} Recency (10), MonetaryMax (11), CustomerTenureDays (7), FirstPurchaseDaysAgo (7), UniqueProducts (13), UniqueDescriptions (10), TotalTransactions (11)
\end{itemize}

\subsection{Visualisations}

\begin{figure}[H]
\centering
\includegraphics[width=0.8\textwidth]{histograms.png}
\caption{Distribution des variables RFM et de l'âge}
\end{figure}

\begin{figure}[H]
\centering
\includegraphics[width=0.8\textwidth]{boxplots.png}
\caption{Détection des outliers sur les variables monétaires}
\end{figure}

\begin{figure}[H]
\centering
\includegraphics[width=0.8\textwidth]{correlations.png}
\caption{Matrice de corrélation}
\end{figure}

\textbf{Corrélations majeures :}
\begin{itemize}
    \item Recency $\leftrightarrow$ Churn : $r = 0,86$ (forte)
    \item Frequency $\leftrightarrow$ MonetaryTotal : $r = 0,57$ (modérée)
\end{itemize}

\begin{figure}[H]
\centering
\includegraphics[width=0.7\textwidth]{scatter_rfm.png}
\caption{Relation Fréquence vs Montant}
\end{figure}

\begin{figure}[H]
\centering
\includegraphics[width=0.7\textwidth]{rfm_segment.png}
\caption{Distribution des segments RFM}
\end{figure}

\begin{table}[h]
\centering
\begin{tabular}{lccc}
\toprule
\textbf{Segment} & \textbf{Effectif} & \textbf{\%} & \textbf{Profil} \\
\midrule
Potentiels & 1 589 & 36,3\% & À activer \\
Fidèles & 1 177 & 26,9\% & Cœur de cible \\
Champions & 979 & 22,4\% & VIP \\
Dormants & 627 & 14,3\% & Risque départ \\
\bottomrule
\end{tabular}
\end{table}

% =============================================================================
% SECTION 4: PREPROCESSING (AJOUT IMPORTANT)
% =============================================================================

\section{Préparation des Données (Préprocessing)}

\subsection{Étape 1 : Parsing des Dates (Guide \S 7)}

\textbf{Problème :} Formats inconsistants (\texttt{17/07/10}, \texttt{2010-10-04}, \texttt{12/09/2009})

\textbf{Méthode :} \texttt{pd.to\_datetime(dayfirst=True, errors='coerce')}

\textbf{Résultat :} 4 nouvelles features : RegYear, RegMonth, RegDay, RegWeekday

\subsection{Étape 2 : Imputation des Valeurs Manquantes (Guide \S 6)}

\begin{table}[h]
\centering
\begin{tabular}{llc}
\toprule
\textbf{Feature} & \textbf{Méthode} & \textbf{Valeur} \\
\midrule
Age (30\% manquants) & Médiane & 49,0 ans \\
AvgDaysBetweenPurchases & Médiane & 25,4 jours \\
SupportTicketsCount (-1, 999) & Médiane après nettoyage & 2,0 \\
SatisfactionScore (-1, 99) & Médiane après nettoyage & 3,0 \\
Gender (Unknown) & Mode & M \\
Codes 99 (7 colonnes) & Médiane & par colonne \\
Valeurs négatives (-1) & Médiane positive & par colonne \\
\bottomrule
\end{tabular}
\caption{Imputation des valeurs manquantes et aberrantes}
\end{table}

\subsection{Étape 3 : Suppression des Features Inutiles (Guide \S 5)}

\begin{itemize}
    \item \textbf{NewsletterSubscribed} : Supprimée (constante "Yes")
    \item \textbf{CustomerID} : Supprimée (identifiant non prédictif)
\end{itemize}

\subsection{Étape 4 : Extraction LastLoginIP (Guide \S 5 + \S 7)}

\textbf{Méthode :} Extraction du premier octet de l'IP

\textbf{Exemple :} "192.168.1.45" $\rightarrow$ "192" (feature IP\_Prefix)

\subsection{Étape 5 : Multicolinéarité (Guide \S 2)}

\textbf{Méthode :} Matrice de corrélation avec seuil $|r| > 0,8$

\textbf{Vérification :} MonetaryTotal vs MonetaryAvg : $r = 0,73$

\textbf{Décision :} Conservation des deux variables (corrélation $< 0,8$)

\subsection{Étape 6 : Feature Engineering (Guide \S 4)}

3 nouvelles features créées :

$$MonetaryPerDay = \frac{MonetaryTotal}{Recency + 1}$$

$$AvgBasketValue = \frac{MonetaryTotal}{Frequency}$$

$$TenureRatio = \frac{Recency}{CustomerTenureDays + 1}$$

\subsection{Étape 7 : Standardisation (Guide \S 1)}

\textbf{Méthode :} StandardScaler (centrage-réduction)

\textbf{Formule :} $Z = \frac{X - \mu}{\sigma}$

\textbf{Anti Data Leakage :} Fit uniquement sur train, transform sur test

\begin{itemize}
    \item Features standardisées : 48 colonnes
    \item Features binaires conservées : 6 colonnes
    \item Cible préservée : Churn (pas de standardisation)
\end{itemize}

\subsection{Récapitulatif des Transformations}

\begin{table}[h]
\centering
\begin{tabular}{lccl}
\toprule
\textbf{Étape} & \textbf{Entrée} & \textbf{Sortie} & \textbf{Guide} \\
\midrule
Chargement & - & 52 cols & - \\
Parsing dates & 52 & 54 cols & \S 7 \\
Imputation & 54 & 54 cols & \S 6 \\
Suppression & 54 & 52 cols & \S 5 \\
Extraction IP & 52 & 52 cols & \S 5+\S 7 \\
Multicolinéarité & 52 & 52 cols & \S 2 \\
Feature Eng. & 52 & 55 cols & \S 4 \\
Standardisation & 55 & 55 cols & \S 1 \\
\bottomrule
\end{tabular}
\caption{Évolution des dimensions du dataset}
\end{table}

\subsection{Vérification du Déséquilibre}

\begin{itemize}
    \item \textbf{Churn} : 67,2\% fidèles / 32,8\% partis (modéré)
    \item \textbf{AccountStatus} : 93,8\% Active (quasi-constante)
\end{itemize}

% =============================================================================
% SECTION 5: CONCLUSION
% =============================================================================

\section{Conclusion et Perspectives}

\begin{tcolorbox}[title=Bilan]
L'étape d'exploration et de préparation est terminée. Les données sont maintenant :
\begin{itemize}
    \item Nettoyées (valeurs manquantes et aberrantes traitées)
    \item Enrichies (3 nouvelles features métier)
    \item Standardisées (même échelle pour toutes les variables)
\end{itemize}
\end{tcolorbox}

\textbf{Prochaine étape :} Clustering avec K-Means pour identifier les segments clients actionnables.

\end{document}